\begin{comment}

 Licensed to the Apache Software Foundation (ASF) under one
 or more contributor license agreements.  See the NOTICE file
 distributed with this work for additional information
 regarding copyright ownership.  The ASF licenses this file
 to you under the Apache License, Version 2.0 (the
 "License"); you may not use this file except in compliance
 with the License.  You may obtain a copy of the License at

   http://www.apache.org/licenses/LICENSE-2.0

 Unless required by applicable law or agreed to in writing,
 software distributed under the License is distributed on an
 "AS IS" BASIS, WITHOUT WARRANTIES OR CONDITIONS OF ANY
 KIND, either express or implied.  See the License for the
 specific language governing permissions and limitations
 under the License.

\end{comment}

\newcommand{\UnivarScriptName}{\texttt{\tt Univar-Stats.dml}}
\newcommand{\BivarScriptName}{\texttt{\tt bivar-stats.dml}}

\newcommand{\OutputRowIDMinimum}{1}
\newcommand{\OutputRowIDMaximum}{2}
\newcommand{\OutputRowIDRange}{3}
\newcommand{\OutputRowIDMean}{4}
\newcommand{\OutputRowIDVariance}{5}
\newcommand{\OutputRowIDStDeviation}{6}
\newcommand{\OutputRowIDStErrorMean}{7}
\newcommand{\OutputRowIDCoeffVar}{8}
\newcommand{\OutputRowIDQuartiles}{?, 13, ?}
\newcommand{\OutputRowIDMedian}{13}
\newcommand{\OutputRowIDIQMean}{14}
\newcommand{\OutputRowIDSkewness}{9}
\newcommand{\OutputRowIDKurtosis}{10}
\newcommand{\OutputRowIDStErrorSkewness}{11}
\newcommand{\OutputRowIDStErrorCurtosis}{12}
\newcommand{\OutputRowIDNumCategories}{15}
\newcommand{\OutputRowIDMode}{16}
\newcommand{\OutputRowIDNumModes}{17}
\newcommand{\OutputRowText}[1]{\mbox{(output row~{#1})\hspace{0.5pt}:}}

\newcommand{\NameStatR}{Pearson's correlation coefficient}
\newcommand{\NameStatChi}{Pearson's~$\chi^2$}
\newcommand{\NameStatPChi}{$P\textrm{-}$value of Pearson's~$\chi^2$}
\newcommand{\NameStatV}{Cram\'er's~$V$}
\newcommand{\NameStatEta}{Eta statistic}
\newcommand{\NameStatF}{$F$~statistic}
\newcommand{\NameStatRho}{Spearman's rank correlation coefficient}

Descriptive statistics are used to quantitatively describe the main characteristics of the data.
They provide meaningful summaries computed over different observations or data records
collected in a study.  These summaries typically form the basis of the initial data exploration
as part of a more extensive statistical analysis.  Such a quantitative analysis assumes that
every variable (also known as, attribute, feature, or column) in the data has a specific
\emph{level of measurement}~\cite{Stevens1946:scales}.

The measurement level of a variable, often called as {\bf variable type}, can either be
\emph{scale} or \emph{categorical}.  A \emph{scale} variable represents the data measured on
an interval scale or ratio scale.  Examples of scale variables include `Height', `Weight',
`Salary', and `Temperature'.  Scale variables are also referred to as \emph{quantitative}
or \emph{continuous} variables.  In contrast, a \emph{categorical} variable has a fixed
limited number of distinct values or categories.  Examples of categorical variables
include `Gender', `Region', `Hair color', `Zipcode', and `Level of Satisfaction'.
Categorical variables can further be classified into two types, \emph{nominal} and
\emph{ordinal}, depending on whether the categories in the variable can be ordered via an
intrinsic ranking.  For example, there is no meaningful ranking among distinct values in
`Hair color' variable, while the categories in `Level of Satisfaction' can be ranked from
highly dissatisfied to highly satisfied.

The input dataset for descriptive statistics is provided in the form of a matrix, whose
rows are the records (data points) and whose columns are the features (i.e.~variables).
Some scripts allow this matrix to be vertically split into two or three matrices.  Descriptive
statistics are computed over the specified features (columns) in the matrix.  Which
statistics are computed depends on the types of the features.  It is important to keep
in mind the following caveats and restrictions:
\begin{Enumerate}
\item  Given a finite set of data records, i.e.~a \emph{sample}, we take their feature
values and compute their \emph{sample statistics}.  These statistics
will vary from sample to sample even if the underlying distribution of feature values
remains the same.  Sample statistics are accurate for the given sample only.
If the goal is to estimate the \emph{distribution statistics} that are parameters of
the (hypothesized) underlying distribution of the features, the corresponding sample
statistics may sometimes be used as approximations, but their accuracy will vary.
\item  In particular, the accuracy of the estimated distribution statistics will be low
if the number of values in the sample is small.  That is, for small samples, the computed
statistics may depend on the randomness of the individual sample values more than on
the underlying distribution of the features.
\item  The accuracy will also be low if the sample records cannot be assumed mutually
independent and identically distributed (i.i.d.), that is, sampled at random from the
same underlying distribution.  In practice, feature values in one record often depend
on other features and other records, including unknown ones.
\item  Most of the computed statistics will have low estimation accuracy in the presence of
extreme values (outliers) or if the underlying distribution has heavy tails, for example
obeys a power law.  However, a few of the computed statistics, such as the median and
\NameStatRho{}, are \emph{robust} to outliers.
\item  Some sample statistics are reported with their \emph{sample standard errors}
in an attempt to quantify their accuracy as distribution parameter estimators.  But these
sample standard errors, in turn, only estimate the underlying distribution's standard
errors and will have low accuracy for small or \mbox{non-i.i.d.} samples, outliers in samples,
or heavy-tailed distributions.
\item  We assume that the quantitative (scale) feature columns do not contain missing
values, infinite values, \texttt{NaN}s, or coded non-numeric values, unless otherwise
specified.  We assume that each categorical feature column contains positive integers
from 1 to the number of categories; for ordinal features, the natural order on
the integers should coincide with the order on the categories.
\end{Enumerate}

\begin{comment}

 Licensed to the Apache Software Foundation (ASF) under one
 or more contributor license agreements.  See the NOTICE file
 distributed with this work for additional information
 regarding copyright ownership.  The ASF licenses this file
 to you under the Apache License, Version 2.0 (the
 "License"); you may not use this file except in compliance
 with the License.  You may obtain a copy of the License at

   http://www.apache.org/licenses/LICENSE-2.0

 Unless required by applicable law or agreed to in writing,
 software distributed under the License is distributed on an
 "AS IS" BASIS, WITHOUT WARRANTIES OR CONDITIONS OF ANY
 KIND, either express or implied.  See the License for the
 specific language governing permissions and limitations
 under the License.

\end{comment}

\subsection{Univariate Statistics}

\noindent{\bf Description}
\smallskip

\emph{Univariate statistics} are the simplest form of descriptive statistics in data
analysis.  They are used to quantitatively describe the main characteristics of each
feature in the data.  For a given dataset matrix, script \UnivarScriptName{} computes
certain univariate statistics for each feature column in the
matrix.  The feature type governs the exact set of statistics computed for that feature.
For example, the statistic \emph{mean} can only be computed on a quantitative (scale)
feature like `Height' and `Temperature'.  It does not make sense to compute the mean
of a categorical attribute like `Hair Color'.


\smallskip
\noindent{\bf Usage}
\smallskip

{\hangindent=\parindent\noindent\it%\tolerance=0
{\tt{}-f } \UnivarScriptName{}
{\tt{} -nvargs}
{\tt{} X=}path/file
{\tt{} TYPES=}path/file
{\tt{} STATS=}path/file
% {\tt{} fmt=}format

}


\medskip
\pagebreak[2]
\noindent{\bf Arguments}
\begin{Description}
\item[{\tt X}:]
Location (on HDFS) to read the data matrix $X$ whose columns we want to
analyze as the features.
\item[{\tt TYPES}:] % (default:\mbox{ }{\tt " "})
Location (on HDFS) to read the single-row matrix whose $i^{\textrm{th}}$
column-cell contains the type of the $i^{\textrm{th}}$ feature column
\texttt{X[,$\,i$]} in the data matrix.  Feature types must be encoded by
integer numbers: $1 = {}$scale, $2 = {}$nominal, $3 = {}$ordinal.
% The default value means ``treat all $X$-columns as scale.''
\item[{\tt STATS}:]
Location (on HDFS) where the output matrix of computed statistics
will be stored.  The format of the output matrix is defined by
Table~\ref{table:univars}.
% \item[{\tt fmt}:] (default:\mbox{ }{\tt "text"})
% Matrix file output format, such as {\tt text}, {\tt mm}, or {\tt csv};
% see read/write functions in SystemDS Language Reference for details.
\end{Description}

\begin{table}[t]\hfil
\begin{tabular}{|rl|c|c|}
\hline
\multirow{2}{*}{Row}& \multirow{2}{*}{Name of Statistic} & \multicolumn{2}{c|}{Applies to:} \\
                            &                            & Scale & Categ.\\
\hline
\OutputRowIDMinimum         & Minimum                    &   +   &       \\
\OutputRowIDMaximum         & Maximum                    &   +   &       \\
\OutputRowIDRange           & Range                      &   +   &       \\
\OutputRowIDMean            & Mean                       &   +   &       \\
\OutputRowIDVariance        & Variance                   &   +   &       \\
\OutputRowIDStDeviation     & Standard deviation         &   +   &       \\
\OutputRowIDStErrorMean     & Standard error of mean     &   +   &       \\
\OutputRowIDCoeffVar        & Coefficient of variation   &   +   &       \\
\OutputRowIDSkewness        & Skewness                   &   +   &       \\
\OutputRowIDKurtosis        & Kurtosis                   &   +   &       \\
\OutputRowIDStErrorSkewness & Standard error of skewness &   +   &       \\
\OutputRowIDStErrorCurtosis & Standard error of kurtosis &   +   &       \\
\OutputRowIDMedian          & Median                     &   +   &       \\
\OutputRowIDIQMean          & Inter quartile mean        &   +   &       \\
\OutputRowIDNumCategories   & Number of categories       &       &   +   \\
\OutputRowIDMode            & Mode                       &       &   +   \\
\OutputRowIDNumModes        & Number of modes            &       &   +   \\
\hline
\end{tabular}\hfil
\caption{The output matrix of \UnivarScriptName{} has one row per each
univariate statistic and one column per input feature.  This table lists
the meaning of each row.  Signs ``+'' show applicability to scale or/and
to categorical features.}
\label{table:univars}
\end{table}


\pagebreak[1]

\smallskip
\noindent{\bf Details}
\smallskip

Given an input matrix \texttt{X}, this script computes the set of all
relevant univariate statistics for each feature column \texttt{X[,$\,i$]}
in~\texttt{X}.  The list of statistics to be computed depends on the
\emph{type}, or \emph{measurement level}, of each column.
The \textrm{TYPES} command-line argument points to a vector containing
the types of all columns.  The types must be provided as per the following
convention: $1 = {}$scale, $2 = {}$nominal, $3 = {}$ordinal.

Below we list all univariate statistics computed by script \UnivarScriptName.
The statistics are collected by relevance into several groups, namely: central
tendency, dispersion, shape, and categorical measures.  The first three groups
contain statistics computed for a quantitative (also known as: numerical, scale,
or continuous) feature; the last group contains the statistics for a categorical
(either nominal or ordinal) feature.  

Let~$n$ be the number of data records (rows) with feature values.
In what follows we fix a column index \texttt{idx} and consider
sample statistics of feature column \texttt{X[}$\,$\texttt{,}$\,$\texttt{idx]}.
Let $v = (v_1, v_2, \ldots, v_n)$ be the values of \texttt{X[}$\,$\texttt{,}$\,$\texttt{idx]}
in their original unsorted order: $v_i = \texttt{X[}i\texttt{,}\,\texttt{idx]}$.
Let $v^s = (v^s_1, v^s_2, \ldots, v^s_n)$ be the same values in the sorted order,
preserving duplicates: $v^s_1 \leq v^s_2 \leq \ldots \leq v^s_n$.

\paragraph{Central tendency measures.}
Sample statistics that describe the location of the quantitative (scale) feature distribution,
represent it with a single value.
\begin{Description}
%%%%%%%%%%%%%%%%%%%% DESCRIPTIVE STATISTIC %%%%%%%%%%%%%%%%%%%%
\item[\it Mean]
\OutputRowText{\OutputRowIDMean}
The arithmetic average over a sample of a quantitative feature.
Computed as the ratio between the sum of values and the number of values:
$\left(\sum_{i=1}^n v_i\right)\!/n$.
Example: the mean of sample $\{$2.2, 3.2, 3.7, 4.4, 5.3, 5.7, 6.1, 6.4, 7.2, 7.8$\}$
equals~5.2.

Note that the mean is significantly affected by extreme values in the sample
and may be misleading as a central tendency measure if the feature varies on
exponential scale.  For example, the mean of $\{$0.01, 0.1, 1.0, 10.0, 100.0$\}$
is 22.222, greater than all the sample values except the~largest.
%%%%%%%%%%%%%%%%%%%%%%%%%%%%%%%%%%%%%%%%%%%%%%%%%%%%%%%%%%%%%%%

\begin{figure}[t]
\setlength{\unitlength}{10pt}
\begin{picture}(33,12)
\put( 6.2, 0.0){\small 2.2}
\put(10.2, 0.0){\small 3.2}
\put(12.2, 0.0){\small 3.7}
\put(15.0, 0.0){\small 4.4}
\put(18.6, 0.0){\small 5.3}
\put(20.2, 0.0){\small 5.7}
\put(21.75,0.0){\small 6.1}
\put(23.05,0.0){\small 6.4}
\put(26.2, 0.0){\small 7.2}
\put(28.6, 0.0){\small 7.8}
\put( 0.5, 0.7){\small 0.0}
\put( 0.1, 3.2){\small 0.25}
\put( 0.5, 5.7){\small 0.5}
\put( 0.1, 8.2){\small 0.75}
\put( 0.5,10.7){\small 1.0}
\linethickness{1.5pt}
\put( 2.0, 1.0){\line(1,0){4.8}}
\put( 6.8, 1.0){\line(0,1){1.0}}
\put( 6.8, 2.0){\line(1,0){4.0}}
\put(10.8, 2.0){\line(0,1){1.0}}
\put(10.8, 3.0){\line(1,0){2.0}}
\put(12.8, 3.0){\line(0,1){1.0}}
\put(12.8, 4.0){\line(1,0){2.8}}
\put(15.6, 4.0){\line(0,1){1.0}}
\put(15.6, 5.0){\line(1,0){3.6}}
\put(19.2, 5.0){\line(0,1){1.0}}
\put(19.2, 6.0){\line(1,0){1.6}}
\put(20.8, 6.0){\line(0,1){1.0}}
\put(20.8, 7.0){\line(1,0){1.6}}
\put(22.4, 7.0){\line(0,1){1.0}}
\put(22.4, 8.0){\line(1,0){1.2}}
\put(23.6, 8.0){\line(0,1){1.0}}
\put(23.6, 9.0){\line(1,0){3.2}}
\put(26.8, 9.0){\line(0,1){1.0}}
\put(26.8,10.0){\line(1,0){2.4}}
\put(29.2,10.0){\line(0,1){1.0}}
\put(29.2,11.0){\line(1,0){4.8}}
\linethickness{0.3pt}
\put( 6.8, 1.0){\circle*{0.3}}
\put(10.8, 1.0){\circle*{0.3}}
\put(12.8, 1.0){\circle*{0.3}}
\put(15.6, 1.0){\circle*{0.3}}
\put(19.2, 1.0){\circle*{0.3}}
\put(20.8, 1.0){\circle*{0.3}}
\put(22.4, 1.0){\circle*{0.3}}
\put(23.6, 1.0){\circle*{0.3}}
\put(26.8, 1.0){\circle*{0.3}}
\put(29.2, 1.0){\circle*{0.3}}
\put( 6.8, 1.0){\vector(1,0){27.2}}
\put( 2.0, 1.0){\vector(0,1){10.8}}
\put( 2.0, 3.5){\line(1,0){10.8}}
\put( 2.0, 6.0){\line(1,0){17.2}}
\put( 2.0, 8.5){\line(1,0){21.6}}
\put( 2.0,11.0){\line(1,0){27.2}}
\put(12.8, 1.0){\line(0,1){2.0}}
\put(19.2, 1.0){\line(0,1){5.0}}
\put(20.0, 1.0){\line(0,1){5.0}}
\put(23.6, 1.0){\line(0,1){7.0}}
\put( 9.0, 4.0){\line(1,0){3.8}}
\put( 9.2, 2.7){\vector(0,1){0.8}}
\put( 9.2, 4.8){\vector(0,-1){0.8}}
\put(19.4, 8.0){\line(1,0){3.0}}
\put(19.6, 7.2){\vector(0,1){0.8}}
\put(19.6, 9.3){\vector(0,-1){0.8}}
\put(13.0, 2.2){\small $q_{25\%}$}
\put(17.3, 2.2){\small $q_{50\%}$}
\put(23.8, 2.2){\small $q_{75\%}$}
\put(20.15,3.5){\small $\mu$}
\put( 8.0, 3.75){\small $\phi_1$}
\put(18.35,7.8){\small $\phi_2$}
\end{picture}
\label{fig:example_quartiles}
\caption{The computation of quartiles, median, and interquartile mean from the
empirical distribution function of the 10-point
sample $\{$2.2, 3.2, 3.7, 4.4, 5.3, 5.7, 6.1, 6.4, 7.2, 7.8$\}$.  Each vertical step in
the graph has height~$1{/}n = 0.1$.  Values $q_{25\%}$, $q_{50\%}$, and $q_{75\%}$ denote
the $1^{\textrm{st}}$, $2^{\textrm{nd}}$, and $3^{\textrm{rd}}$ quartiles correspondingly;
value~$\mu$ denotes the median.  Values $\phi_1$ and $\phi_2$ show the partial contribution
of border points (quartiles) $v_3=3.7$ and $v_8=6.4$ into the interquartile mean.}
\end{figure}

%%%%%%%%%%%%%%%%%%%% DESCRIPTIVE STATISTIC %%%%%%%%%%%%%%%%%%%%
\item[\it Median]
\OutputRowText{\OutputRowIDMedian}
The ``middle'' value that separates the higher half of the sample values
(in a sorted order) from the lower half.
To compute the median, we sort the sample in the increasing order, preserving
duplicates: $v^s_1 \leq v^s_2 \leq \ldots \leq v^s_n$.
If $n$ is odd, the median equals $v^s_i$ where $i = (n\,{+}\,1)\,{/}\,2$,
same as the $50^{\textrm{th}}$~percentile of the sample.
If $n$ is even, there are two ``middle'' values $v^s_{n/2}$ and $v^s_{n/2\,+\,1}$,
so we compute the median as the mean of these two values.
(For even~$n$ we compute the $50^{\textrm{th}}$~percentile as~$v^s_{n/2}$,
not as the median.)  Example: the median of sample
$\{$2.2, 3.2, 3.7, 4.4, 5.3, 5.7, 6.1, 6.4, 7.2, 7.8$\}$
equals $(5.3\,{+}\,5.7)\,{/}\,2$~${=}$~5.5, see Figure~\ref{fig:example_quartiles}.

Unlike the mean, the median is not sensitive to extreme values in the sample,
i.e.\ it is robust to outliers.  It works better as a measure of central tendency
for heavy-tailed distributions and features that vary on exponential scale.
However, the median is sensitive to small sample size.
%%%%%%%%%%%%%%%%%%%% DESCRIPTIVE STATISTIC %%%%%%%%%%%%%%%%%%%%
\item[\it Interquartile mean]
\OutputRowText{\OutputRowIDIQMean}
For a sample of a quantitative feature, this is
the mean of the values greater than or equal to the $1^{\textrm{st}}$ quartile
and less than or equal the $3^{\textrm{rd}}$ quartile.
In other words, it is a ``truncated mean'' where the lowest 25$\%$ and
the highest 25$\%$ of the sorted values are omitted in its computation.
The two ``border values'', i.e.\ the $1^{\textrm{st}}$ and the $3^{\textrm{rd}}$
quartiles themselves, contribute to this mean only partially.
This measure is occasionally used as the ``robust'' version of the mean
that is less sensitive to the extreme values.

To compute the measure, we sort the sample in the increasing order,
preserving duplicates: $v^s_1 \leq v^s_2 \leq \ldots \leq v^s_n$.
We set $j = \lceil n{/}4 \rceil$ for the $1^{\textrm{st}}$ quartile index
and $k = \lceil 3n{/}4 \rceil$ for the $3^{\textrm{rd}}$ quartile index,
then compute the following weighted mean:
\begin{equation*}
\frac{1}{3{/}4 - 1{/}4} \left[
\left(\frac{j}{n} - \frac{1}{4}\right) v^s_j \,\,+ 
\sum_{j<i<k} \left(\frac{i}{n} - \frac{i\,{-}\,1}{n}\right) v^s_i 
\,\,+\,\, \left(\frac{3}{4} - \frac{k\,{-}\,1}{n}\right) v^s_k\right]
\end{equation*}
In other words, all sample values between the $1^{\textrm{st}}$ and the $3^{\textrm{rd}}$
quartile enter the sum with weights $2{/}n$, times their number of duplicates, while the
two quartiles themselves enter the sum with reduced weights.  The weights are proportional
to the vertical steps in the empirical distribution function of the sample, see
Figure~\ref{fig:example_quartiles} for an illustration.
Example: the interquartile mean of sample
$\{$2.2, 3.2, 3.7, 4.4, 5.3, 5.7, 6.1, 6.4, 7.2, 7.8$\}$ equals the sum
$0.1 (3.7\,{+}\,6.4) + 0.2 (4.4\,{+}\,5.3\,{+}\,5.7\,{+}\,6.1)$,
which equals~5.31.
\end{Description}


\paragraph{Dispersion measures.}
Statistics that describe the amount of variation or spread in a quantitative
(scale) data feature.
\begin{Description}
%%%%%%%%%%%%%%%%%%%% DESCRIPTIVE STATISTIC %%%%%%%%%%%%%%%%%%%%
\item[\it Variance]
\OutputRowText{\OutputRowIDVariance}
A measure of dispersion, or spread-out, of sample values around their mean,
expressed in units that are the square of those of the feature itself.
Computed as the sum of squared differences between the values
in the sample and their mean, divided by one less than the number of
values: $\sum_{i=1}^n (v_i - \bar{v})^2\,/\,(n\,{-}\,1)$ where 
$\bar{v}=\left(\sum_{i=1}^n v_i\right)\!/n$.
Example: the variance of sample
$\{$2.2, 3.2, 3.7, 4.4, 5.3, 5.7, 6.1, 6.4, 7.2, 7.8$\}$ equals~3.24.
Note that at least two values ($n\geq 2$) are required to avoid division
by zero.  Sample variance is sensitive to outliers, even more than the mean.
%%%%%%%%%%%%%%%%%%%% DESCRIPTIVE STATISTIC %%%%%%%%%%%%%%%%%%%%
\item[\it Standard deviation]
\OutputRowText{\OutputRowIDStDeviation}
A measure of dispersion around the mean, the square root of variance.
Computed by taking the square root of the sample variance;
see \emph{Variance} above on computing the variance.
Example: the standard deviation of sample
$\{$2.2, 3.2, 3.7, 4.4, 5.3, 5.7, 6.1, 6.4, 7.2, 7.8$\}$ equals~1.8.
At least two values are required to avoid division by zero.
Note that standard deviation is sensitive to outliers.  

Standard deviation is used in conjunction with the mean to determine
an interval containing a given percentage of the feature values,
assuming the normal distribution.  In a large sample from a normal
distribution, around 68\% of the cases fall within one standard
deviation and around 95\% of cases fall within two standard deviations
of the mean.  For example, if the mean age is 45 with a standard deviation
of 10, around 95\% of the cases would be between 25 and 65 in a normal
distribution.
%%%%%%%%%%%%%%%%%%%% DESCRIPTIVE STATISTIC %%%%%%%%%%%%%%%%%%%%
\item[\it Coefficient of variation]
\OutputRowText{\OutputRowIDCoeffVar}
The ratio of the standard deviation to the mean, i.e.\ the
\emph{relative} standard deviation, of a quantitative feature sample.
Computed by dividing the sample \emph{standard deviation} by the
sample \emph{mean}, see above for their computation details.
Example: the coefficient of variation for sample
$\{$2.2, 3.2, 3.7, 4.4, 5.3, 5.7, 6.1, 6.4, 7.2, 7.8$\}$
equals 1.8$\,{/}\,$5.2~${\approx}$~0.346.

This metric is used primarily with non-negative features such as
financial or population data.  It is sensitive to outliers.
Note: zero mean causes division by zero, returning infinity or \texttt{NaN}.
At least two values (records) are required to compute the standard deviation.
%%%%%%%%%%%%%%%%%%%% DESCRIPTIVE STATISTIC %%%%%%%%%%%%%%%%%%%%
\item[\it Minimum]
\OutputRowText{\OutputRowIDMinimum}
The smallest value of a quantitative sample, computed as $\min v = v^s_1$.
Example: the minimum of sample
$\{$2.2, 3.2, 3.7, 4.4, 5.3, 5.7, 6.1, 6.4, 7.2, 7.8$\}$
equals~2.2.
%%%%%%%%%%%%%%%%%%%% DESCRIPTIVE STATISTIC %%%%%%%%%%%%%%%%%%%%
\item[\it Maximum]
\OutputRowText{\OutputRowIDMaximum}
The largest value of a quantitative sample, computed as $\max v = v^s_n$.
Example: the maximum of sample
$\{$2.2, 3.2, 3.7, 4.4, 5.3, 5.7, 6.1, 6.4, 7.2, 7.8$\}$
equals~7.8.
%%%%%%%%%%%%%%%%%%%% DESCRIPTIVE STATISTIC %%%%%%%%%%%%%%%%%%%%
\item[\it Range]
\OutputRowText{\OutputRowIDRange}
The difference between the largest and the smallest value of a quantitative
sample, computed as $\max v - \min v = v^s_n - v^s_1$.
It provides information about the overall spread of the sample values.
Example: the range of sample
$\{$2.2, 3.2, 3.7, 4.4, 5.3, 5.7, 6.1, 6.4, 7.2, 7.8$\}$
equals 7.8$\,{-}\,$2.2~${=}$~5.6.
%%%%%%%%%%%%%%%%%%%% DESCRIPTIVE STATISTIC %%%%%%%%%%%%%%%%%%%%
\item[\it Standard error of the mean]
\OutputRowText{\OutputRowIDStErrorMean}
A measure of how much the value of the sample mean may vary from sample
to sample taken from the same (hypothesized) distribution of the feature.
It helps to roughly bound the distribution mean, i.e.\
the limit of the sample mean as the sample size tends to infinity.
Under certain assumptions (e.g.\ normality and large sample), the difference
between the distribution mean and the sample mean is unlikely to exceed
2~standard errors.

The measure is computed by dividing the sample standard deviation
by the square root of the number of values~$n$; see \emph{standard deviation}
for its computation details.  Ensure $n\,{\geq}\,2$ to avoid division by~0.
Example: for sample
$\{$2.2, 3.2, 3.7, 4.4, 5.3, 5.7, 6.1, 6.4, 7.2, 7.8$\}$
with the mean of~5.2 the standard error of the mean
equals 1.8$\,{/}\sqrt{10}$~${\approx}$~0.569.

Note that the standard error itself is subject to sample randomness.
Its accuracy as an error estimator may be low if the sample size is small
or \mbox{non-i.i.d.}, if there are outliers, or if the distribution has
heavy tails.
%%%%%%%%%%%%%%%%%%%% DESCRIPTIVE STATISTIC %%%%%%%%%%%%%%%%%%%%
% \item[\it Quartiles]
% \OutputRowText{\OutputRowIDQuartiles}
% %%% dsDefn %%%%
% The values of a quantitative feature
% that divide an ordered/sorted set of data records into four equal-size groups.
% The $1^{\textrm{st}}$ quartile, or the $25^{\textrm{th}}$ percentile, splits
% the sorted data into the lowest $25\%$ and the highest~$75\%$.  In other words,
% it is the middle value between the minimum and the median.  The $2^{\textrm{nd}}$
% quartile is the median itself, the value that separates the higher half of
% the data (in the sorted order) from the lower half.  Finally, the $3^{\textrm{rd}}$
% quartile, or the $75^{\textrm{th}}$ percentile, divides the sorted data into
% lowest $75\%$ and highest~$25\%$.\par
% %%% dsComp %%%%
% To compute the quartiles for a data column \texttt{X[,i]} with $n$ numerical values
% we sort it in the increasing order, preserving duplicates, then return 
% \texttt{X}${}^{\textrm{sort}}$\texttt{[}$k$\texttt{,i]}
% where $k = \lceil pn \rceil$ for $p = 0.25$, $0.5$, and~$0.75$.
% When $n$ is even, the $2^{\textrm{nd}}$ quartile (the median) is further adjusted
% to equal the mean of two middle values
% $\texttt{X}^{\textrm{sort}}\texttt{[}n{/}2\texttt{,i]}$ and
% $\texttt{X}^{\textrm{sort}}\texttt{[}n{/}2\,{+}\,1\texttt{,i]}$.
% %%% dsWarn %%%%
% We assume that the feature column does not contain \texttt{NaN}s or coded non-numeric values.
% %%% dsExmpl %%%
% \textbf{Example(s).}
\end{Description}


\paragraph{Shape measures.}
Statistics that describe the shape and symmetry of the quantitative (scale)
feature distribution estimated from a sample of its values.
\begin{Description}
%%%%%%%%%%%%%%%%%%%% DESCRIPTIVE STATISTIC %%%%%%%%%%%%%%%%%%%%
\item[\it Skewness]
\OutputRowText{\OutputRowIDSkewness}
It measures how symmetrically the values of a feature are spread out
around the mean.  A significant positive skewness implies a longer (or fatter)
right tail, i.e. feature values tend to lie farther away from the mean on the
right side.  A significant negative skewness implies a longer (or fatter) left
tail.  The normal distribution is symmetric and has a skewness value of~0;
however, its sample skewness is likely to be nonzero, just close to zero.
As a guideline, a skewness value more than twice its standard error is taken
to indicate a departure from symmetry.

Skewness is computed as the $3^{\textrm{rd}}$~central moment divided by the cube
of the standard deviation.  We estimate the $3^{\textrm{rd}}$~central moment as
the sum of cubed differences between the values in the feature column and their
sample mean, divided by the number of values:  
$\sum_{i=1}^n (v_i - \bar{v})^3 / n$
where $\bar{v}=\left(\sum_{i=1}^n v_i\right)\!/n$.
The standard deviation is computed
as described above in \emph{standard deviation}.  To avoid division by~0,
at least two different sample values are required.  Example: for sample
$\{$2.2, 3.2, 3.7, 4.4, 5.3, 5.7, 6.1, 6.4, 7.2, 7.8$\}$
with the mean of~5.2 and the standard deviation of~1.8
skewness is estimated as $-1.0728\,{/}\,1.8^3 \approx -0.184$.
Note: skewness is sensitive to outliers.
%%%%%%%%%%%%%%%%%%%% DESCRIPTIVE STATISTIC %%%%%%%%%%%%%%%%%%%%
\item[\it Standard error in skewness]
\OutputRowText{\OutputRowIDStErrorSkewness}
A measure of how much the sample skewness may vary from sample to sample,
assuming that the feature is normally distributed, which makes its
distribution skewness equal~0.  
Given the number~$n$ of sample values, the standard error is computed as
\begin{equation*}
\sqrt{\frac{6n\,(n-1)}{(n-2)(n+1)(n+3)}}
\end{equation*}
This measure can tell us, for example:
\begin{Itemize}
\item If the sample skewness lands within two standard errors from~0, its
positive or negative sign is non-significant, may just be accidental.
\item If the sample skewness lands outside this interval, the feature
is unlikely to be normally distributed.
\end{Itemize}
At least 3~values ($n\geq 3$) are required to avoid arithmetic failure.
Note that the standard error is inaccurate if the feature distribution is
far from normal or if the number of samples is small.
%%%%%%%%%%%%%%%%%%%% DESCRIPTIVE STATISTIC %%%%%%%%%%%%%%%%%%%%
\item[\it Kurtosis]
\OutputRowText{\OutputRowIDKurtosis}
As a distribution parameter, kurtosis is a measure of the extent to which
feature values cluster around a central point.  In other words, it quantifies
``peakedness'' of the distribution: how tall and sharp the central peak is
relative to a standard bell curve.

Positive kurtosis (\emph{leptokurtic} distribution) indicates that, relative
to a normal distribution:
\begin{Itemize}
\item observations cluster more about the center (peak-shaped),
\item the tails are thinner at non-extreme values, 
\item the tails are thicker at extreme values.
\end{Itemize}
Negative kurtosis (\emph{platykurtic} distribution) indicates that, relative
to a normal distribution:
\begin{Itemize}
\item observations cluster less about the center (box-shaped),
\item the tails are thicker at non-extreme values, 
\item the tails are thinner at extreme values.
\end{Itemize}
Kurtosis of a normal distribution is zero; however, the sample kurtosis
(computed here) is likely to deviate from zero.

Sample kurtosis is computed as the $4^{\textrm{th}}$~central moment divided
by the $4^{\textrm{th}}$~power of the standard deviation, minus~3.
We estimate the $4^{\textrm{th}}$~central moment as the sum of the
$4^{\textrm{th}}$~powers of differences between the values in the feature column
and their sample mean, divided by the number of values:
$\sum_{i=1}^n (v_i - \bar{v})^4 / n$
where $\bar{v}=\left(\sum_{i=1}^n v_i\right)\!/n$.
The standard deviation is computed as described above, see \emph{standard deviation}.

Note that kurtosis is sensitive to outliers, and requires at least two different
sample values.  Example: for sample
$\{$2.2, 3.2, 3.7, 4.4, 5.3, 5.7, 6.1, 6.4, 7.2, 7.8$\}$
with the mean of~5.2 and the standard deviation of~1.8,
sample kurtosis equals $16.6962\,{/}\,1.8^4 - 3 \approx -1.41$.
%%%%%%%%%%%%%%%%%%%% DESCRIPTIVE STATISTIC %%%%%%%%%%%%%%%%%%%%
\item[\it Standard error in kurtosis]
\OutputRowText{\OutputRowIDStErrorCurtosis}
A measure of how much the sample kurtosis may vary from sample to sample,
assuming that the feature is normally distributed, which makes its
distribution kurtosis equal~0.
Given the number~$n$ of sample values, the standard error is computed as
\begin{equation*}
\sqrt{\frac{24n\,(n-1)^2}{(n-3)(n-2)(n+3)(n+5)}}
\end{equation*}
This measure can tell us, for example:
\begin{Itemize}
\item If the sample kurtosis lands within two standard errors from~0, its
positive or negative sign is non-significant, may just be accidental.
\item If the sample kurtosis lands outside this interval, the feature
is unlikely to be normally distributed.
\end{Itemize}
At least 4~values ($n\geq 4$) are required to avoid arithmetic failure.
Note that the standard error is inaccurate if the feature distribution is
far from normal or if the number of samples is small.
\end{Description}


\paragraph{Categorical measures.}  Statistics that describe the sample of
a categorical feature, either nominal or ordinal.  We represent all
categories by integers from~1 to the number of categories; we call
these integers \emph{category~IDs}.
\begin{Description}
%%%%%%%%%%%%%%%%%%%% DESCRIPTIVE STATISTIC %%%%%%%%%%%%%%%%%%%%
\item[\it Number of categories]
\OutputRowText{\OutputRowIDNumCategories}
The maximum category~ID that occurs in the sample.  Note that some
categories with~IDs \emph{smaller} than this maximum~ID may have
no~occurrences in the sample, without reducing the number of categories.
However, any categories with~IDs \emph{larger} than the maximum~ID with
no occurrences in the sample will not be counted.
Example: in sample $\{$1, 3, 3, 3, 3, 4, 4, 5, 7, 7, 7, 7, 8, 8, 8$\}$
the number of categories is reported as~8.  Category~IDs 2 and~6, which have
zero occurrences, are still counted; but if there is a category with
ID${}=9$ and zero occurrences, it is not counted.
%%%%%%%%%%%%%%%%%%%% DESCRIPTIVE STATISTIC %%%%%%%%%%%%%%%%%%%%
\item[\it Mode]
\OutputRowText{\OutputRowIDMode}
The most frequently occurring category value.
If several values share the greatest frequency of occurrence, then each
of them is a mode; but here we report only the smallest of these modes.
Example: in sample $\{$1, 3, 3, 3, 3, 4, 4, 5, 7, 7, 7, 7, 8, 8, 8$\}$
the modes are 3 and~7, with 3 reported.

Computed by counting the number of occurrences for each category,
then taking the smallest category~ID that has the maximum count.
Note that the sample modes may be different from the distribution modes,
i.e.\ the categories whose (hypothesized) underlying probability is the
maximum over all categories.
%%%%%%%%%%%%%%%%%%%% DESCRIPTIVE STATISTIC %%%%%%%%%%%%%%%%%%%%
\item[\it Number of modes]
\OutputRowText{\OutputRowIDNumModes}
The number of category values that each have the largest frequency
count in the sample.  
Example: in sample $\{$1, 3, 3, 3, 3, 4, 4, 5, 7, 7, 7, 7, 8, 8, 8$\}$
there are two category IDs (3 and~7) that occur the maximum count of 4~times;
hence, we return~2.

Computed by counting the number of occurrences for each category,
then counting how many categories have the maximum count.
Note that the sample modes may be different from the distribution modes,
i.e.\ the categories whose (hypothesized) underlying probability is the
maximum over all categories.
\end{Description}


\smallskip
\noindent{\bf Returns}
\smallskip

The output matrix containing all computed statistics is of size $17$~rows and
as many columns as in the input matrix~\texttt{X}.  Each row corresponds to
a particular statistic, according to the convention specified in
Table~\ref{table:univars}.  The first $14$~statistics are applicable for
\emph{scale} columns, and the last $3$~statistics are applicable for categorical,
i.e.\ nominal and ordinal, columns.


\pagebreak[2]

\smallskip
\noindent{\bf Examples}
\smallskip

{\hangindent=\parindent\noindent\tt
\hml -f \UnivarScriptName{} -nvargs X=/user/biadmin/X.mtx
  TYPES=/user/biadmin/types.mtx
  STATS=/user/biadmin/stats.mtx

}


\begin{comment}

 Licensed to the Apache Software Foundation (ASF) under one
 or more contributor license agreements.  See the NOTICE file
 distributed with this work for additional information
 regarding copyright ownership.  The ASF licenses this file
 to you under the Apache License, Version 2.0 (the
 "License"); you may not use this file except in compliance
 with the License.  You may obtain a copy of the License at

   http://www.apache.org/licenses/LICENSE-2.0

 Unless required by applicable law or agreed to in writing,
 software distributed under the License is distributed on an
 "AS IS" BASIS, WITHOUT WARRANTIES OR CONDITIONS OF ANY
 KIND, either express or implied.  See the License for the
 specific language governing permissions and limitations
 under the License.

\end{comment}

\subsection{Bivariate Statistics}

\noindent{\bf Description}
\smallskip

Bivariate statistics are used to quantitatively describe the association between
two features, such as test their statistical (in-)dependence or measure
the accuracy of one data feature predicting the other feature, in a sample.
The \BivarScriptName{} script computes common bivariate statistics,
such as \NameStatR{} and \NameStatChi{}, in parallel for many pairs
of data features.  For a given dataset matrix, script \BivarScriptName{} computes
certain bivariate statistics for the given feature (column) pairs in the
matrix.  The feature types govern the exact set of statistics computed for that pair.
For example, \NameStatR{} can only be computed on two quantitative (scale)
features like `Height' and `Temperature'. 
It does not make sense to compute the linear correlation of two categorical attributes
like `Hair Color'. 


\smallskip
\noindent{\bf Usage}
\smallskip

{\hangindent=\parindent\noindent\it%\tolerance=0
{\tt{}-f }path/\/\BivarScriptName{}
{\tt{} -nvargs}
{\tt{} X=}path/file
{\tt{} index1=}path/file
{\tt{} index2=}path/file
{\tt{} types1=}path/file
{\tt{} types2=}path/file
{\tt{} OUTDIR=}path
% {\tt{} fmt=}format

}


\smallskip
\noindent{\bf Arguments}
\begin{Description}
\item[{\tt X}:]
Location (on HDFS) to read the data matrix $X$ whose columns are the features
that we want to compare and correlate with bivariate statistics.
\item[{\tt index1}:] % (default:\mbox{ }{\tt " "})
Location (on HDFS) to read the single-row matrix that lists the column indices
of the \emph{first-argument} features in pairwise statistics.
Its $i^{\textrm{th}}$ entry (i.e.\ $i^{\textrm{th}}$ column-cell) contains the
index $k$ of column \texttt{X[,$\,k$]} in the data matrix
whose bivariate statistics need to be computed.
% The default value means ``use all $X$-columns from the first to the last.''
\item[{\tt index2}:] % (default:\mbox{ }{\tt " "})
Location (on HDFS) to read the single-row matrix that lists the column indices
of the \emph{second-argument} features in pairwise statistics.
Its $j^{\textrm{th}}$ entry (i.e.\ $j^{\textrm{th}}$ column-cell) contains the
index $l$ of column \texttt{X[,$\,l$]} in the data matrix
whose bivariate statistics need to be computed.
% The default value means ``use all $X$-columns from the first to the last.''
\item[{\tt types1}:] % (default:\mbox{ }{\tt " "})
Location (on HDFS) to read the single-row matrix that lists the \emph{types}
of the \emph{first-argument} features in pairwise statistics.
Its $i^{\textrm{th}}$ entry (i.e.\ $i^{\textrm{th}}$ column-cell) contains the type
of column \texttt{X[,$\,k$]} in the data matrix, where $k$ is the $i^{\textrm{th}}$
entry in the {\tt index1} matrix.  Feature types must be encoded by
integer numbers: $1 = {}$scale, $2 = {}$nominal, $3 = {}$ordinal.
% The default value means ``treat all referenced $X$-columns as scale.''
\item[{\tt types2}:] % (default:\mbox{ }{\tt " "})
Location (on HDFS) to read the single-row matrix that lists the \emph{types}
of the \emph{second-argument} features in pairwise statistics.
Its $j^{\textrm{th}}$ entry (i.e.\ $j^{\textrm{th}}$ column-cell) contains the type
of column \texttt{X[,$\,l$]} in the data matrix, where $l$ is the $j^{\textrm{th}}$
entry in the {\tt index2} matrix.  Feature types must be encoded by
integer numbers: $1 = {}$scale, $2 = {}$nominal, $3 = {}$ordinal.
% The default value means ``treat all referenced $X$-columns as scale.''
\item[{\tt OUTDIR}:]
Location path (on HDFS) where the output matrices with computed bivariate
statistics will be stored.  The matrices' file names and format are defined
in Table~\ref{table:bivars}.
% \item[{\tt fmt}:] (default:\mbox{ }{\tt "text"})
% Matrix file output format, such as {\tt text}, {\tt mm}, or {\tt csv};
% see read/write functions in SystemDS Language Reference for details.
\end{Description}

\begin{table}[t]\hfil
\begin{tabular}{|lll|}
\hline\rule{0pt}{12pt}%
Ouput File / Matrix         & Row$\,$\# & Name of Statistic   \\[2pt]
\hline\hline\rule{0pt}{12pt}%
\emph{All Files}            &     1     & 1-st feature column \\
\rule{1em}{0pt}"            &     2     & 2-nd feature column \\[2pt]
\hline\rule{0pt}{12pt}%
bivar.scale.scale.stats     &     3     & \NameStatR          \\[2pt]
\hline\rule{0pt}{12pt}%
bivar.nominal.nominal.stats &     3     & \NameStatChi        \\
\rule{1em}{0pt}"            &     4     & Degrees of freedom  \\
\rule{1em}{0pt}"            &     5     & \NameStatPChi       \\
\rule{1em}{0pt}"            &     6     & \NameStatV          \\[2pt]
\hline\rule{0pt}{12pt}%
bivar.nominal.scale.stats   &     3     & \NameStatEta        \\
\rule{1em}{0pt}"            &     4     & \NameStatF          \\[2pt]
\hline\rule{0pt}{12pt}%
bivar.ordinal.ordinal.stats &     3     & \NameStatRho        \\[2pt]
\hline
\end{tabular}\hfil
\caption{%
The output matrices of \BivarScriptName{} have one row per one bivariate
statistic and one column per one pair of input features.  This table lists
the meaning of each matrix and each row.%
% Signs ``+'' show applicability to scale or/and to categorical features.
}
\label{table:bivars}
\end{table}



\pagebreak[2]

\noindent{\bf Details}
\smallskip

Script \BivarScriptName{} takes an input matrix \texttt{X} whose columns represent
the features and whose rows represent the records of a data sample.
Given \texttt{X}, the script computes certain relevant bivariate statistics
for specified pairs of feature columns \texttt{X[,$\,i$]} and \texttt{X[,$\,j$]}.
Command-line parameters \texttt{index1} and \texttt{index2} specify the files with
column pairs of interest to the user.  Namely, the file given by \texttt{index1}
contains the vector of the 1st-attribute column indices and the file given
by \texttt{index2} has the vector of the 2nd-attribute column indices, with
``1st'' and ``2nd'' referring to their places in bivariate statistics.
Note that both \texttt{index1} and \texttt{index2} files should contain a 1-row matrix
of positive integers.

The bivariate statistics to be computed depend on the \emph{types}, or
\emph{measurement levels}, of the two columns.
The types for each pair are provided in the files whose locations are specified by
\texttt{types1} and \texttt{types2} command-line parameters.
These files are also 1-row matrices, i.e.\ vectors, that list the 1st-attribute and
the 2nd-attribute column types in the same order as their indices in the
\texttt{index1} and \texttt{index2} files.  The types must be provided as per
the following convention: $1 = {}$scale, $2 = {}$nominal, $3 = {}$ordinal.

The script orgainizes its results into (potentially) four output matrices, one per
each type combination.  The types of bivariate statistics are defined using the types
of the columns that were used for their arguments, with ``ordinal'' sometimes
retrogressing to ``nominal.''  Table~\ref{table:bivars} describes what each column
in each output matrix contains.  In particular, the script includes the following
statistics:
\begin{Itemize}
\item For a pair of scale (quantitative) columns, \NameStatR;
\item For a pair of nominal columns (with finite-sized, fixed, unordered domains), 
the \NameStatChi{} and its p-value;
\item For a pair of one scale column and one nominal column, \NameStatF{};
\item For a pair of ordinal columns (ordered domains depicting ranks), \NameStatRho.
\end{Itemize}
Note that, as shown in Table~\ref{table:bivars}, the output matrices contain the
column indices of the features involved in each statistic.
Moreover, if the output matrix does not contain
a value in a certain cell then it should be interpreted as a~$0$
(sparse matrix representation).

Below we list all bivariate statistics computed by script \BivarScriptName.
The statistics are collected into several groups by the type of their input
features.  We refer to the two input features as $v_1$ and $v_2$ unless
specified otherwise; the value pairs are $(v_{1,i}, v_{2,i})$ for $i=1,\ldots,n$,
where $n$ is the number of rows in \texttt{X}, i.e.\ the sample size.


\paragraph{Scale-vs-scale statistics.}
Sample statistics that describe association between two quantitative (scale) features.
A scale feature has numerical values, with the natural ordering relation.
\begin{Description}
%%%%%%%%%%%%%%%%%%%% DESCRIPTIVE STATISTIC %%%%%%%%%%%%%%%%%%%%
\item[\it\NameStatR]:
A measure of linear dependence between two numerical features:
\begin{equation*}
r \,\,=\,\, \frac{\Cov(v_1, v_2)}{\sqrt{\Var v_1 \Var v_2}}
\,\,=\,\, \frac{\sum_{i=1}^n (v_{1,i} - \bar{v}_1) (v_{2,i} - \bar{v}_2)}%
{\sqrt{\sum_{i=1}^n (v_{1,i} - \bar{v}_1)^{2\mathstrut} \cdot \sum_{i=1}^n (v_{2,i} - \bar{v}_2)^{2\mathstrut}}}
\end{equation*}
Commonly denoted by~$r$, correlation ranges between $-1$ and $+1$, reaching ${\pm}1$ when all value
pairs $(v_{1,i}, v_{2,i})$ lie on the same line.  Correlation near~0 means that a line is not a good
way to represent the dependence between the two features; however, this does not imply independence.
The sign indicates direction of the linear association: $r > 0$ ($r < 0$) if one feature tends to
linearly increase (decrease) when the other feature increases.  Nonlinear association, if present,
may disobey this sign.
\NameStatR{} is symmetric: $r(v_1, v_2) = r(v_2, v_1)$; it does not change if we transform $v_1$ and $v_2$
to $a + b v_1$ and $c + d v_2$ where $a, b, c, d$ are constants and $b, d > 0$.

Suppose that we use simple linear regression to represent one feature given the other, say
represent $v_{2,i} \approx \alpha + \beta v_{1,i}$ by selecting $\alpha$ and $\beta$
to minimize the least-squares error $\sum_{i=1}^n (v_{2,i} - \alpha - \beta v_{1,i})^2$.
Then the best error equals
\begin{equation*}
\min_{\alpha, \beta} \,\,\sum_{i=1}^n \big(v_{2,i} - \alpha - \beta v_{1,i}\big)^2 \,\,=\,\,
(1 - r^2) \,\sum_{i=1}^n \big(v_{2,i} - \bar{v}_2\big)^2
\end{equation*}
In other words, $1\,{-}\,r^2$ is the ratio of the residual sum of squares to
the total sum of squares.  Hence, $r^2$ is an accuracy measure of the linear regression.
\end{Description}


\paragraph{Nominal-vs-nominal statistics.}
Sample statistics that describe association between two nominal categorical features.
Both features' value domains are encoded with positive integers in arbitrary order:
nominal features do not order their value domains.
\begin{Description}
%%%%%%%%%%%%%%%%%%%% DESCRIPTIVE STATISTIC %%%%%%%%%%%%%%%%%%%%
\item[\it\NameStatChi]:
A measure of how much the frequencies of value pairs of two categorical features deviate from
statistical independence.  Under independence, the probability of every value pair must equal
the product of probabilities of each value in the pair:
$\Prob[a, b] - \Prob[a]\,\Prob[b] = 0$.  But we do not know these (hypothesized) probabilities;
we only know the sample frequency counts.  Let $n_{a,b}$ be the frequency count of pair
$(a, b)$, let $n_a$ and $n_b$ be the frequency counts of $a$~alone and of $b$~alone.  Under
independence, difference $n_{a,b}{/}n - (n_a{/}n)(n_b{/}n)$ is unlikely to be exactly~0 due
to sample randomness, yet it is unlikely to be too far from~0.  For some pairs $(a,b)$ it may
deviate from~0 farther than for other pairs.  \NameStatChi{}~is an aggregate measure that
combines squares of these differences across all value pairs:
\begin{equation*}
\chi^2 \,\,=\,\, \sum_{a,\,b} \Big(\frac{n_a n_b}{n}\Big)^{-1} \Big(n_{a,b} - \frac{n_a n_b}{n}\Big)^2
\,=\,\, \sum_{a,\,b} \frac{(O_{a,b} - E_{a,b})^2}{E_{a,b}}
\end{equation*}
where $O_{a,b} = n_{a,b}$ are the \emph{observed} frequencies and $E_{a,b} = (n_a n_b){/}n$ are
the \emph{expected} frequencies for all pairs~$(a,b)$.  Under independence (plus other standard
assumptions) the sample~$\chi^2$ closely follows a well-known distribution, making it a basis for
statistical tests for independence, see~\emph{\NameStatPChi} for details.  Note that \NameStatChi{}
does \emph{not} measure the strength of dependence: even very weak dependence may result in a
significant deviation from independence if the counts are large enough.  Use~\NameStatV{} instead
to measure the strength of dependence.
%%%%%%%%%%%%%%%%%%%% DESCRIPTIVE STATISTIC %%%%%%%%%%%%%%%%%%%%
\item[\it Degrees of freedom]:
An integer parameter required for the interpretation of~\NameStatChi{} measure.  Under independence
(plus other standard assumptions) the sample~$\chi^2$ statistic is approximately distributed as the
sum of $d$~squares of independent normal random variables with mean~0 and variance~1, where $d$ is
this integer parameter.  For a pair of categorical features such that the $1^{\textrm{st}}$~feature
has $k_1$ categories and the $2^{\textrm{nd}}$~feature has $k_2$ categories, the number of degrees
of freedom is $d = (k_1 - 1)(k_2 - 1)$.
%%%%%%%%%%%%%%%%%%%% DESCRIPTIVE STATISTIC %%%%%%%%%%%%%%%%%%%%
\item[\it\NameStatPChi]:
A measure of how likely we would observe the current frequencies of value pairs of two categorical
features assuming their statistical independence.  More precisely, it computes the probability that
the sum of $d$~squares of independent normal random variables with mean~0 and variance~1
(called the $\chi^2$~distribution with $d$ degrees of freedom) generates a value at least as large
as the current sample \NameStatChi.  The $d$ parameter is \emph{degrees of freedom}, see above.
Under independence (plus other standard assumptions) the sample \NameStatChi{} closely follows the
$\chi^2$~distribution and is unlikely to land very far into its tail.  On the other hand, if the
two features are dependent, their sample \NameStatChi{} becomes arbitrarily large as $n\to\infty$
and lands extremely far into the tail of the $\chi^2$~distribution given a large enough data sample.
\NameStatPChi{} returns the tail ``weight'' on the right-hand side of \NameStatChi:
\begin{equation*}
P\,\,=\,\, \Prob\big[r \geq \textrm{\NameStatChi} \,\,\big|\,\, r \sim \textrm{the $\chi^2$ distribution}\big]
\end{equation*}
As any probability, $P$ ranges between 0 and~1.  If $P\leq 0.05$, the dependence between the two
features may be considered statistically significant (i.e.\ their independence is considered
statistically ruled out).  For highly dependent features, it is not unusual to have $P\leq 10^{-20}$
or less, in which case our script will simply return $P = 0$.  Independent features should have
their $P\geq 0.05$ in about 95\% cases.
%%%%%%%%%%%%%%%%%%%% DESCRIPTIVE STATISTIC %%%%%%%%%%%%%%%%%%%%
\item[\it\NameStatV]:
A measure for the strength of association, i.e.\ of statistical dependence, between two categorical
features, conceptually similar to \NameStatR.  It divides the observed~\NameStatChi{} by the maximum
possible~$\chi^2_{\textrm{max}}$ given $n$ and the number $k_1, k_2$~of categories in each feature,
then takes the square root.  Thus, \NameStatV{} ranges from 0 to~1,
where 0 implies no association and 1 implies the maximum possible association (one-to-one
correspondence) between the two features.  See \emph{\NameStatChi} for the computation of~$\chi^2$;
its maximum${} = {}$%
$n\cdot\min\{k_1\,{-}\,1, k_2\,{-}\,1\}$ where the $1^{\textrm{st}}$~feature
has $k_1$ categories and the $2^{\textrm{nd}}$~feature has $k_2$ categories~\cite{AcockStavig1979:CramersV},
so
\begin{equation*}
\textrm{\NameStatV} \,\,=\,\, \sqrt{\frac{\textrm{\NameStatChi}}{n\cdot\min\{k_1\,{-}\,1, k_2\,{-}\,1\}}}
\end{equation*}
As opposed to \NameStatPChi, which goes to~0 (rapidly) as the features' dependence increases,
\NameStatV{} goes towards~1 (slowly) as the dependence increases.  Both \NameStatChi{} and
\NameStatPChi{} are very sensitive to~$n$, but in \NameStatV{} this is mitigated by taking the
ratio.
\end{Description}


\paragraph{Nominal-vs-scale statistics.}
Sample statistics that describe association between a categorical feature
(order ignored) and a quantitative (scale) feature.
The values of the categorical feature must be coded as positive integers.
\begin{Description}
%%%%%%%%%%%%%%%%%%%% DESCRIPTIVE STATISTIC %%%%%%%%%%%%%%%%%%%%
\item[\it\NameStatEta]:
A measure for the strength of association (statistical dependence) between a nominal feature
and a scale feature, conceptually similar to \NameStatR.  Ranges from 0 to~1, approaching 0
when there is no association and approaching 1 when there is a strong association.  
The nominal feature, treated as the independent variable, is assumed to have relatively few
possible values, all with large frequency counts.  The scale feature is treated as the dependent
variable.  Denoting the nominal feature by~$x$ and the scale feature by~$y$, we have:
\begin{equation*}
\eta^2 \,=\, 1 - \frac{\sum_{i=1}^{n} \big(y_i - \hat{y}[x_i]\big)^2}{\sum_{i=1}^{n} (y_i - \bar{y})^2},
\,\,\,\,\textrm{where}\,\,\,\,
\hat{y}[x] = \frac{1}{\mathop{\mathrm{freq}}(x)}\sum_{i=1}^n  
\,\left\{\!\!\begin{array}{rl} y_i & \textrm{if $x_i = x$}\\ 0 & \textrm{otherwise}\end{array}\right.\!\!\!
\end{equation*}
and $\bar{y} = (1{/}n)\sum_{i=1}^n y_i$ is the mean.  Value $\hat{y}[x]$ is the average 
of~$y_i$ among all records where $x_i = x$; it can also be viewed as the ``predictor'' 
of $y$ given~$x$.  Then $\sum_{i=1}^{n} (y_i - \hat{y}[x_i])^2$ is the residual error
sum-of-squares and $\sum_{i=1}^{n} (y_i - \bar{y})^2$ is the total sum-of-squares for~$y$. 
Hence, $\eta^2$ measures the accuracy of predicting $y$ with~$x$, just like the
``R-squared'' statistic measures the accuracy of linear regression.  Our output $\eta$
is the square root of~$\eta^2$.
%%%%%%%%%%%%%%%%%%%% DESCRIPTIVE STATISTIC %%%%%%%%%%%%%%%%%%%%
\item[\it\NameStatF]:
A measure of how much the values of the scale feature, denoted here by~$y$,
deviate from statistical independence on the nominal feature, denoted by~$x$.
The same measure appears in the one-way analysis of vari\-ance (ANOVA).
Like \NameStatChi, \NameStatF{} is used to test the hypothesis that
$y$~is independent from~$x$, given the following assumptions:
\begin{Itemize}
\item The scale feature $y$ has approximately normal distribution whose mean
may depend only on~$x$ and variance is the same for all~$x$;
\item The nominal feature $x$ has relatively small value domain with large
frequency counts, the $x_i$-values are treated as fixed (non-random);
\item All records are sampled independently of each other.
\end{Itemize}
To compute \NameStatF{}, we first compute $\hat{y}[x]$ as the average of~$y_i$
among all records where $x_i = x$.  These $\hat{y}[x]$ can be viewed as
``predictors'' of $y$ given~$x$; if $y$ is independent on~$x$, they should
``predict'' only the global mean~$\bar{y}$.  Then we form two sums-of-squares:
\begin{Itemize}
\item \emph{Residual} sum-of-squares of the ``predictor'' accuracy: $y_i - \hat{y}[x_i]$;
\item \emph{Explained} sum-of-squares of the ``predictor'' variability: $\hat{y}[x_i] - \bar{y}$.
\end{Itemize}
\NameStatF{} is the ratio of the explained sum-of-squares to
the residual sum-of-squares, each divided by their corresponding degrees
of freedom:
\begin{equation*}
F \,\,=\,\, 
\frac{\sum_{x}\, \mathop{\mathrm{freq}}(x) \, \big(\hat{y}[x] - \bar{y}\big)^2 \,\big/\,\, (k\,{-}\,1)}%
{\sum_{i=1}^{n} \big(y_i - \hat{y}[x_i]\big)^2 \,\big/\,\, (n\,{-}\,k)} \,\,=\,\,
\frac{n\,{-}\,k}{k\,{-}\,1} \cdot \frac{\eta^2}{1 - \eta^2}
\end{equation*}
Here $k$ is the domain size of the nominal feature~$x$.  The $k$ ``predictors''
lose 1~freedom due to their linear dependence with~$\bar{y}$; similarly,
the $n$~$y_i$-s lose $k$~freedoms due to the ``predictors''.

The statistic can test if the independence hypothesis of $y$ from $x$ is reasonable;
more generally (with relaxed normality assumptions) it can test the hypothesis that
\emph{the mean} of $y$ among records with a given~$x$ is the same for all~$x$.
Under this hypothesis \NameStatF{} has, or approximates, the $F(k\,{-}\,1, n\,{-}\,k)$-distribution.
But if the mean of $y$ given $x$ depends on~$x$, \NameStatF{}
becomes arbitrarily large as $n\to\infty$ (with $k$~fixed) and lands extremely far
into the tail of the $F(k\,{-}\,1, n\,{-}\,k)$-distribution given a large enough data sample.
\end{Description}


\paragraph{Ordinal-vs-ordinal statistics.}
Sample statistics that describe association between two ordinal categorical features.
Both features' value domains are encoded with positive integers, so that the natural
order of the integers coincides with the order in each value domain.
\begin{Description}
%%%%%%%%%%%%%%%%%%%% DESCRIPTIVE STATISTIC %%%%%%%%%%%%%%%%%%%%
\item[\it\NameStatRho]:
A measure for the strength of association (statistical dependence) between
two ordinal features, conceptually similar to \NameStatR.  Specifically, it is \NameStatR{}
applied to the feature vectors in which all values are replaced by their ranks, i.e.\ 
their positions if the vector is sorted.  The ranks of identical (duplicate) values
are replaced with their average rank.  For example, in vector
$(15, 11, 26, 15, 8)$ the value ``15'' occurs twice with ranks 3 and~4 per the sorted
order $(8_1, 11_2, 15_3, 15_4, 26_5)$; so, both values are assigned their average
rank of $3.5 = (3\,{+}\,4)\,{/}\,2$ and the vector is replaced by~$(3.5,\, 2,\, 5,\, 3.5,\, 1)$.

Our implementation of \NameStatRho{} is geared towards features having small value domains
and large counts for the values.  Given the two input vectors, we form a contingency table $T$
of pairwise frequency counts, as well as a vector of frequency counts for each feature: $f_1$
and~$f_2$.  Here in $T_{i,j}$, $f_{1,i}$, $f_{2,j}$ indices $i$ and~$j$ refer to the
order-preserving integer encoding of the feature values.
We use prefix sums over $f_1$ and~$f_2$ to compute the values' average ranks:
$r_{1,i} = \sum_{j=1}^{i-1} f_{1,j} + (f_{1,i}\,{+}\,1){/}2$, and analogously for~$r_2$.
Finally, we compute rank variances for $r_1, r_2$ weighted by counts $f_1, f_2$ and their
covariance weighted by~$T$, before applying the standard formula for \NameStatR:
\begin{equation*}
\rho \,\,=\,\, \frac{\Cov_T(r_1, r_2)}{\sqrt{\Var_{f_1}(r_1)\Var_{f_2}(r_2)}}
\,\,=\,\, \frac{\sum_{i,j} T_{i,j} (r_{1,i} - \bar{r}_1) (r_{2,j} - \bar{r}_2)}%
{\sqrt{\sum_i f_{1,i} (r_{1,i} - \bar{r}_1)^{2\mathstrut} \cdot \sum_j f_{2,j} (r_{2,j} - \bar{r}_2)^{2\mathstrut}}}
\end{equation*}
where $\bar{r}_1 = \sum_i r_{1,i} f_{1,i}{/}n$, analogously for~$\bar{r}_2$.
The value of $\rho$ lies between $-1$ and $+1$, with sign indicating the prevalent direction
of the association: $\rho > 0$ ($\rho < 0$) means that one feature tends to increase (decrease)
when the other feature increases.  The correlation becomes~1 when the two features are
monotonically related.
\end{Description}


\smallskip
\noindent{\bf Returns}
\smallskip

A collection of (potentially) 4 matrices.  Each matrix contains bivariate statistics that
resulted from a different combination of feature types.  There is one matrix for scale-scale
statistics (which includes \NameStatR), one for nominal-nominal statistics (includes \NameStatChi{}),
one for nominal-scale statistics (includes \NameStatF) and one for ordinal-ordinal statistics
(includes \NameStatRho).  If any of these matrices is not produced, then no pair of columns required
the corresponding type combination.  See Table~\ref{table:bivars} for the matrix naming and
format details.


\smallskip
\pagebreak[2]

\noindent{\bf Examples}
\smallskip

{\hangindent=\parindent\noindent\tt
\hml -f \BivarScriptName{} -nvargs
X=/user/biadmin/X.mtx 
index1=/user/biadmin/S1.mtx 
index2=/user/biadmin/S2.mtx 
types1=/user/biadmin/K1.mtx 
types2=/user/biadmin/K2.mtx 
OUTDIR=/user/biadmin/stats.mtx

}



\begin{comment}

 Licensed to the Apache Software Foundation (ASF) under one
 or more contributor license agreements.  See the NOTICE file
 distributed with this work for additional information
 regarding copyright ownership.  The ASF licenses this file
 to you under the Apache License, Version 2.0 (the
 "License"); you may not use this file except in compliance
 with the License.  You may obtain a copy of the License at

   http://www.apache.org/licenses/LICENSE-2.0

 Unless required by applicable law or agreed to in writing,
 software distributed under the License is distributed on an
 "AS IS" BASIS, WITHOUT WARRANTIES OR CONDITIONS OF ANY
 KIND, either express or implied.  See the License for the
 specific language governing permissions and limitations
 under the License.

\end{comment}

\subsection{Stratified Bivariate Statistics}

\noindent{\bf Description}
\smallskip

The {\tt stratstats.dml} script computes common bivariate statistics, such
as correlation, slope, and their p-value, in parallel for many pairs of input
variables in the presence of a confounding categorical variable.  The values
of this confounding variable group the records into strata (subpopulations),
in which all bivariate pairs are assumed free of confounding.  The script
uses the same data model as in one-way analysis of covariance (ANCOVA), with
strata representing population samples.  It also outputs univariate stratified
and bivariate unstratified statistics.

\begin{table}[t]\hfil
\begin{tabular}{|l|ll|ll|ll||ll|}
\hline
Month of the year & \multicolumn{2}{l|}{October} & \multicolumn{2}{l|}{November} &
    \multicolumn{2}{l||}{December} & \multicolumn{2}{c|}{Oct$\,$--$\,$Dec} \\
Customers, millions    & 0.6 & 1.4 & 1.4 & 0.6 & 3.0 & 1.0 & 5.0 & 3.0 \\
\hline
Promotion (0 or 1)     & 0   & 1   & 0   & 1   & 0   & 1   & 0   & 1   \\
Avg.\ sales per 1000   & 0.4 & 0.5 & 0.9 & 1.0 & 2.5 & 2.6 & 1.8 & 1.3 \\
\hline
\end{tabular}\hfil
\caption{Stratification example: the effect of the promotion on average sales
becomes reversed and amplified (from $+0.1$ to $-0.5$) if we ignore the months.}
\label{table:stratexample}
\end{table}

To see how data stratification mitigates confounding, consider an (artificial)
example in Table~\ref{table:stratexample}.  A highly seasonal retail item
was marketed with and without a promotion over the final 3~months of the year.
In each month the sale was more likely with the promotion than without it.
But during the peak holiday season, when shoppers came in greater numbers and
bought the item more often, the promotion was less frequently used.  As a result,
if the 4-th quarter data is pooled together, the promotion's effect becomes
reversed and magnified.  Stratifying by month restores the positive correlation.

The script computes its statistics in parallel over all possible pairs from two
specified sets of covariates.  The 1-st covariate is a column in input matrix~$X$
and the 2-nd covariate is a column in input matrix~$Y$; matrices $X$ and~$Y$ may
be the same or different.  The columns of interest are given by their index numbers
in special matrices.  The stratum column, specified in its own matrix, is the same
for all covariate pairs.

Both covariates in each pair must be numerical, with the 2-nd covariate normally
distributed given the 1-st covariate (see~Details).  Missing covariate values or
strata are represented by~``NaN''.  Records with NaN's are selectively omitted
wherever their NaN's are material to the output statistic.

\smallskip
\pagebreak[3]

\noindent{\bf Usage}
\smallskip

{\hangindent=\parindent\noindent\it%
{\tt{}-f }path/\/{\tt{}stratstats.dml}
{\tt{} -nvargs}
{\tt{} X=}path/file
{\tt{} Xcid=}path/file
{\tt{} Y=}path/file
{\tt{} Ycid=}path/file
{\tt{} S=}path/file
{\tt{} Scid=}int
{\tt{} O=}path/file
{\tt{} fmt=}format

}


\smallskip
\noindent{\bf Arguments}
\begin{Description}
\item[{\tt X}:]
Location (on HDFS) to read matrix $X$ whose columns we want to use as
the 1-st covariate (i.e.~as the feature variable)
\item[{\tt Xcid}:] (default:\mbox{ }{\tt " "})
Location to read the single-row matrix that lists all index numbers
of the $X$-columns used as the 1-st covariate; the default value means
``use all $X$-columns''
\item[{\tt Y}:] (default:\mbox{ }{\tt " "})
Location to read matrix $Y$ whose columns we want to use as the 2-nd
covariate (i.e.~as the response variable); the default value means
``use $X$ in place of~$Y$''
\item[{\tt Ycid}:] (default:\mbox{ }{\tt " "})
Location to read the single-row matrix that lists all index numbers
of the $Y$-columns used as the 2-nd covariate; the default value means
``use all $Y$-columns''
\item[{\tt S}:] (default:\mbox{ }{\tt " "})
Location to read matrix $S$ that has the stratum column.
Note: the stratum column must contain small positive integers; all fractional
values are rounded; stratum IDs of value ${\leq}\,0$ or NaN are treated as
missing.  The default value for {\tt S} means ``use $X$ in place of~$S$''
\item[{\tt Scid}:] (default:\mbox{ }{\tt 1})
The index number of the stratum column in~$S$
\item[{\tt O}:]
Location to store the output matrix defined in Table~\ref{table:stratoutput}
\item[{\tt fmt}:] (default:\mbox{ }{\tt "text"})
Matrix file output format, such as {\tt text}, {\tt mm}, or {\tt csv};
see read/write functions in SystemML Language Reference for details.
\end{Description}


\begin{table}[t]\small\hfil
\begin{tabular}{|rcl|rcl|}
\hline
& Col.\# & Meaning & & Col.\# & Meaning \\
\hline
\multirow{9}{*}{\begin{sideways}1-st covariate\end{sideways}}\hspace{-1em}
& 01     & $X$-column number                & 
\multirow{9}{*}{\begin{sideways}2-nd covariate\end{sideways}}\hspace{-1em}
& 11     & $Y$-column number                \\
& 02     & presence count for $x$           & 
& 12     & presence count for $y$           \\
& 03     & global mean $(x)$                & 
& 13     & global mean $(y)$                \\
& 04     & global std.\ dev. $(x)$          & 
& 14     & global std.\ dev. $(y)$          \\
& 05     & stratified std.\ dev. $(x)$      & 
& 15     & stratified std.\ dev. $(y)$      \\
& 06     & $R^2$ for $x \sim {}$strata      & 
& 16     & $R^2$ for $y \sim {}$strata      \\
& 07     & adjusted $R^2$ for $x \sim {}$strata      & 
& 17     & adjusted $R^2$ for $y \sim {}$strata      \\
& 08     & p-value, $x \sim {}$strata       & 
& 18     & p-value, $y \sim {}$strata       \\
& 09--10 & reserved                         & 
& 19--20 & reserved                         \\
\hline
\multirow{9}{*}{\begin{sideways}$y\sim x$, NO strata\end{sideways}}\hspace{-1.15em}
& 21     & presence count $(x, y)$          &
\multirow{10}{*}{\begin{sideways}$y\sim x$ AND strata$\!\!\!\!$\end{sideways}}\hspace{-1.15em}
& 31     & presence count $(x, y, s)$       \\
& 22     & regression slope                 &
& 32     & regression slope                 \\
& 23     & regres.\ slope std.\ dev.        &
& 33     & regres.\ slope std.\ dev.        \\
& 24     & correlation${} = \pm\sqrt{R^2}$  &
& 34     & correlation${} = \pm\sqrt{R^2}$  \\
& 25     & residual std.\ dev.              &
& 35     & residual std.\ dev.              \\
& 26     & $R^2$ in $y$ due to $x$          &
& 36     & $R^2$ in $y$ due to $x$          \\
& 27     & adjusted $R^2$ in $y$ due to $x$ &
& 37     & adjusted $R^2$ in $y$ due to $x$ \\
& 28     & p-value for ``slope = 0''        &
& 38     & p-value for ``slope = 0''        \\
& 29     & reserved                         &
& 39     & \# strata with ${\geq}\,2$ count \\
& 30     & reserved                         &
& 40     & reserved                         \\
\hline
\end{tabular}\hfil
\caption{The {\tt stratstats.dml} output matrix has one row per each distinct
pair of 1-st and 2-nd covariates, and 40 columns with the statistics described
here.}
\label{table:stratoutput}
\end{table}




\noindent{\bf Details}
\smallskip

Suppose we have $n$ records of format $(i, x, y)$, where $i\in\{1,\ldots, k\}$ is
a stratum number and $(x, y)$ are two numerical covariates.  We want to analyze
conditional linear relationship between $y$ and $x$ conditioned by~$i$.
Note that $x$, but not~$y$, may represent a categorical variable if we assign a
numerical value to each category, for example 0 and 1 for two categories.

We assume a linear regression model for~$y$:
\begin{equation}
y_{i,j} \,=\, \alpha_i + \beta x_{i,j} + \eps_{i,j}\,, \quad\textrm{where}\,\,\,\,
\eps_{i,j} \sim \Normal(0, \sigma^2)
\label{eqn:stratlinmodel}
\end{equation}
Here $i = 1\ldots k$ is a stratum number and $j = 1\ldots n_i$ is a record number
in stratum~$i$; by $n_i$ we denote the number of records available in stratum~$i$.
The noise term~$\eps_{i,j}$ is assumed to have the same variance in all strata.
When $n_i\,{>}\,0$, we can estimate the means of $x_{i, j}$ and $y_{i, j}$ in
stratum~$i$ as
\begin{equation*}
\bar{x}_i \,= \Big(\sum\nolimits_{j=1}^{n_i} \,x_{i, j}\Big) / n_i\,;\quad
\bar{y}_i \,= \Big(\sum\nolimits_{j=1}^{n_i} \,y_{i, j}\Big) / n_i
\end{equation*}
If $\beta$ is known, the best estimate for $\alpha_i$ is $\bar{y}_i - \beta \bar{x}_i$,
which gives the prediction error sum-of-squares of
\begin{equation}
\sum\nolimits_{i=1}^k \sum\nolimits_{j=1}^{n_i} \big(y_{i,j} - \beta x_{i,j} - (\bar{y}_i - \beta \bar{x}_i)\big)^2
\,\,=\,\, \beta^{2\,}V_x \,-\, 2\beta \,V_{x,y} \,+\, V_y
\label{eqn:stratsumsq}
\end{equation}
where $V_x$, $V_y$, and $V_{x, y}$ are, correspondingly, the ``stratified'' sample
estimates of variance $\Var(x)$ and $\Var(y)$ and covariance $\Cov(x,y)$ computed as
\begin{align*}
V_x     \,&=\, \sum\nolimits_{i=1}^k \sum\nolimits_{j=1}^{n_i} \big(x_{i,j} - \bar{x}_i\big)^2; \quad
V_y     \,=\, \sum\nolimits_{i=1}^k \sum\nolimits_{j=1}^{n_i} \big(y_{i,j} - \bar{y}_i\big)^2;\\
V_{x,y} \,&=\, \sum\nolimits_{i=1}^k \sum\nolimits_{j=1}^{n_i} \big(x_{i,j} - \bar{x}_i\big)\big(y_{i,j} - \bar{y}_i\big)
\end{align*}
They are stratified because we compute the sample (co-)variances in each stratum~$i$
separately, then combine by summation.  The stratified estimates for $\Var(X)$ and $\Var(Y)$
tend to be smaller than the non-stratified ones (with the global mean instead of $\bar{x}_i$
and~$\bar{y}_i$) since $\bar{x}_i$ and $\bar{y}_i$ fit closer to $x_{i,j}$ and $y_{i,j}$
than the global means.  The stratified variance estimates the uncertainty in $x_{i,j}$ 
and~$y_{i,j}$ given their stratum~$i$.

Minimizing over~$\beta$ the error sum-of-squares~(\ref{eqn:stratsumsq})
gives us the regression slope estimate \mbox{$\hat{\beta} = V_{x,y} / V_x$},
with~(\ref{eqn:stratsumsq}) becoming the residual sum-of-squares~(RSS):
\begin{equation*}
\mathrm{RSS} \,\,=\, \,
\sum\nolimits_{i=1}^k \sum\nolimits_{j=1}^{n_i} \big(y_{i,j} - 
\hat{\beta} x_{i,j} - (\bar{y}_i - \hat{\beta} \bar{x}_i)\big)^2
\,\,=\,\,  V_y \,\big(1 \,-\, V_{x,y}^2 / (V_x V_y)\big)
\end{equation*}
The quantity $\hat{R}^2 = V_{x,y}^2 / (V_x V_y)$, called \emph{$R$-squared}, estimates the fraction
of stratified variance in~$y_{i,j}$ explained by covariate $x_{i, j}$ in the linear 
regression model~(\ref{eqn:stratlinmodel}).  We define \emph{stratified correlation} as the
square root of~$\hat{R}^2$ taken with the sign of~$V_{x,y}$.  We also use RSS to estimate
the residual standard deviation $\sigma$ in~(\ref{eqn:stratlinmodel}) that models the prediction error
of $y_{i,j}$ given $x_{i,j}$ and the stratum:
\begin{equation*}
\hat{\beta}\, =\, \frac{V_{x,y}}{V_x}; \,\,\,\, \hat{R} \,=\, \frac{V_{x,y}}{\sqrt{V_x V_y}};
\,\,\,\, \hat{R}^2 \,=\, \frac{V_{x,y}^2}{V_x V_y};
\,\,\,\, \hat{\sigma} \,=\, \sqrt{\frac{\mathrm{RSS}}{n - k - 1}}\,\,\,\,
\Big(n = \sum_{i=1}^k n_i\Big)
\end{equation*}

The $t$-test and the $F$-test for the null-hypothesis of ``$\beta = 0$'' are
obtained by considering the effect of $\hat{\beta}$ on the residual sum-of-squares,
measured by the decrease from $V_y$ to~RSS.
The $F$-statistic is the ratio of the ``explained'' sum-of-squares
to the residual sum-of-squares, divided by their corresponding degrees of freedom.
There are $n\,{-}\,k$ degrees of freedom for~$V_y$, parameter $\beta$ reduces that
to $n\,{-}\,k\,{-}\,1$ for~RSS, and their difference $V_y - {}$RSS has just 1 degree
of freedom:
\begin{equation*}
F \,=\, \frac{(V_y - \mathrm{RSS})/1}{\mathrm{RSS}/(n\,{-}\,k\,{-}\,1)}
\,=\, \frac{\hat{R}^2\,(n\,{-}\,k\,{-}\,1)}{1-\hat{R}^2}; \quad
t \,=\, \hat{R}\, \sqrt{\frac{n\,{-}\,k\,{-}\,1}{1-\hat{R}^2}}.
\end{equation*}
The $t$-statistic is simply the square root of the $F$-statistic with the appropriate
choice of sign.  If the null hypothesis and the linear model are both true, the $t$-statistic
has Student $t$-distribution with $n\,{-}\,k\,{-}\,1$ degrees of freedom.  We can
also compute it if we divide $\hat{\beta}$ by its estimated standard deviation:
\begin{equation*}
\stdev(\hat{\beta})_{\mathrm{est}} \,=\, \hat{\sigma}\,/\sqrt{V_x} \quad\Longrightarrow\quad
t \,=\, \hat{R}\sqrt{V_y} \,/\, \hat{\sigma} \,=\, \beta \,/\, \stdev(\hat{\beta})_{\mathrm{est}}
\end{equation*}
The standard deviation estimate for~$\beta$ is included in {\tt stratstats.dml} output.

\smallskip
\noindent{\bf Returns}
\smallskip

The output matrix format is defined in Table~\ref{table:stratoutput}.

\smallskip
\noindent{\bf Examples}
\smallskip

{\hangindent=\parindent\noindent\tt
\hml -f stratstats.dml -nvargs X=/user/biadmin/X.mtx Xcid=/user/biadmin/Xcid.mtx
  Y=/user/biadmin/Y.mtx Ycid=/user/biadmin/Ycid.mtx S=/user/biadmin/S.mtx Scid=2
  O=/user/biadmin/Out.mtx fmt=csv

}
{\hangindent=\parindent\noindent\tt
\hml -f stratstats.dml -nvargs X=/user/biadmin/Data.mtx Xcid=/user/biadmin/Xcid.mtx
  Ycid=/user/biadmin/Ycid.mtx Scid=7 O=/user/biadmin/Out.mtx

}

%\smallskip
%\noindent{\bf See Also}
%\smallskip
%
%For non-stratified bivariate statistics with a wider variety of input data types
%and statistical tests, see \ldots.  For general linear regression, see
%{\tt LinearRegDS.dml} and {\tt LinearRegCG.dml}.  For logistic regression, appropriate
%when the response variable is categorical, see {\tt MultiLogReg.dml}.


